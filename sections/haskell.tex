\section{Haskell}
\begin{frame}[fragile]
  \frametitle{Haskell Platform}
  El paquete \texttt{haskell-platform} contiene el compilador, depurador, gestor de 
  librerías y otras utilidades para programar en Haskell.
  En otras distribuciones puede instalarse directamente \texttt{ghc} 
  (Glasgow Haskell Compiler):
  \begin{lstlisting}
  sudo apt-get install haskell-platform
  \end{lstlisting}
\end{frame}

\begin{frame}
  \frametitle{El intérprete: GHCi}
  GHC es un compilador de Haskell con GHCi como intérprete asociado. 
  El intérprete permite los siguientes comandos:
  \begin{itemize}
    \item \texttt{:q} \qquad  Quitar
    \item \texttt{:l} \qquad  Cargar módulo
    \item \texttt{:r} \qquad  Recargar módulos
    \item \texttt{:t} \qquad  Consultar tipos
  \end{itemize}
  
  Una vez cargado el intérprete podemos utilizarlo para probar el lenguaje.
  Haskell permite operaciones aritméticas básicas, y operaciones con
  cadenas, listas o booleanos.
\end{frame}

\begin{frame}[fragile]
  Podemos probar el uso de un puñado de funciones simples. Nótese cómo las funciones
  se escriben dejando sus argumentos a su lado y separados por espacios. ¡Estamos usando
  \textbf{notación polaca}!
  
\begin{lstlisting}
succ 9			-- El siguiente de 9
min 28 51		-- El menor entre 28 y 51
(max 6 4) * (succ 9)	-- El producto de lo definido
(*) (max 6 4) (succ 9)	-- Notado como prefijo
id 42			-- Identidad
\end{lstlisting}
\end{frame}