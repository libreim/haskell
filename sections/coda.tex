\section{Más}

\begin{frame}[fragile]{Curry-Howard}
  Como las instancias sólo pueden construirse desde los constructores definidos
  y las funciones no tienen efectos secundarios, podemos razonar la
  corrección del código con inducción estructural:

  \begin{lstlisting}[language=haskell]
qsort []     = []
qsort (x:xs) = qsort [y | y<-xs, y<=x]
            ++ [x]
            ++ qsort [y | y<-xs, y>x]
  \end{lstlisting}

  \textit{Quicksort} funciona porque ordena correctamente una lista vacía y porque, supuesto
  que funcione para listas menores que una dada, funciona para ella.
\end{frame}
