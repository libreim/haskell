%% Paquetes y formato %%
\documentclass{beamer} % Beamer
\usepackage[spanish]{babel} % Traducciones
\usepackage[utf8]{inputenc} % Uso de caracteres UTF-8
\usepackage{amsfonts} %%%%%%%%%%%%%%%%%%%%%%%%%%%%
\usepackage{amsmath}  % Paquetes de matemáticas  %
\usepackage{amssymb}  %%%%%%%%%%%%%%%%%%%%%%%%%%%%
\usepackage{listings} % Coloreado del código

%% Comandos %%
\newcommand{\iffd}{\overset{\Delta}{\iff}} % Si y sólo si por definición
\newcommand{\N}{\mathbb{N}}                % Números naturales
\newcommand{\R}{\mathbb{R}}                % Números reales
\newcommand{\C}{\mathbb{C}}                % Números complejos
\newcommand{\Power}{\mathcal{P}}           % Conjunto potencia
\newcommand{\st}{\mathrel{} : \mathrel{}}  % Tal que (dos puntos)


%% Temas %%	
% Tema y tema de color
\usetheme{Dresden}
\usecolortheme{beaver}

% Color de los enlaces (tex.stackexchange.com/questions/13423)
\definecolor{links}{HTML}{DF0101}
\hypersetup{colorlinks,linkcolor=,urlcolor=links}


%% Título y otros %%
\title{Introducción a Haskell}  % Título
\subtitle{Y a la programación funcional}   % Subtítulo
\author[@pbaeyens \and @M42]    % Autores (tex.stackexchange.com/questions/63259)
{\texorpdfstring{
    \begin{columns}
      \column{.45\linewidth}
      \centering
      Pablo Baeyens\\
      \href{http://www.github.com/pbaeyens}{@pbaeyens}
      \column{.45\linewidth}
      \centering
      Mario Román\\
      \href{http://www.github.com/M42}{@M42}
    \end{columns}
}{Pablo Baeyens \and Mario Román}}
\date{OSL 2015} % Fecha


%% Presentación %%
\begin{document}

  % Página de título
  \frame{\titlepage}

  % Índice
  %\frame{
  %  \frametitle{Índice}
  %  \tableofcontents[currentsection]
  %}

  % Secciones
  \section{Haskell}
  \section{Tipos}
  \section{Funciones}
  \section{Ejemplos}
\begin{frame}[fragile]
      \frametitle{Quicksort}
	Implementación del algoritmo Quicksort
 	\begin{lstlisting}
qsort []     = []
qsort (x:xs) = qsort [y | y<-xs, y<=x] 
            ++ [x]
            ++ qsort [y | y<-xs, y>x]
 	\end{lstlisting}
\end{frame}
  
  % Código fuente
  \frame{
    \frametitle{¡Contribuye!}
    El código fuente de estas diapositivas está disponible en:
    \begin{center}
      \huge \href{http://github.com/M42/osl-talk-haskell}{github.com/M42/osl-talk-haskell}
    \end{center}
    Erratas, correcciones y aportaciones son bienvenidas.
  }


    
\end{document}
